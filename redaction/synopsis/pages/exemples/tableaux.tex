
\section{Les tableaux}


\subsection{tableau normal}

% exemple d'un tableau simple
\begin{table}[H]
	
	\centering
	\caption{Légende du tableau}	
	\label{tab:label_tableau}
	
	\begin{tabular}{|l|l|l|}
		% header
		\hline \bf colonne 1 & \bf colonne 2 & \bf colonne 3 \\
		
		% le contenu
		\hline texte & texte & texte \\
		\hline 
		
	\end{tabular}
	
\end{table}


\subsection{tableau avec du texte long}

\begin{table}[H]
	
	\centering
	\caption{tableau avec du texte long}	
	\label{tab:label_tableau_texte_long}
	
	\begin{tabular}{|p{12cm}|p{3cm}|}
		% header
		\hline \bf colonne 1 & \bf colonne 2 \tabularnewline
		
		% le contenu
		\hline 
			\raggedright \lipsum[1] & 
			\centering texte 
			\tabularnewline
		\hline 
		
	\end{tabular}
	
\end{table}


\subsection{tableau ayant beaucoup de lignes}


% pour utiliser les listes dans le tableau
% \setlist[itemize]{label=$-$,leftmargin=*,parsep=0cm,itemsep=0cm,topsep=0cm}


% tableau de 2 colonnes, avec texte centré	
% longtable pour les long tableaux sur plusieurs pages
%\setlength\LTleft{-1.5cm}		% décaler le tableau sur la gauche
\begin{longtable}{|c|c|}
	
	% entête de la première page
	\hline \bf colonne 1 & \bf colonne 2 \\ 
	\hline
	\endfirsthead
	
	%\hline
	%\endfirstfoot 	
	
	% Entête de toutes les pages	
	%\hline
	%\endhead
	
	% Bas de toutes les pages
	%\hline
	%\endfoot		
	
	% Contenu du tableau 
	\hline 	du texte &	\\ 
	\hline

\end{longtable}