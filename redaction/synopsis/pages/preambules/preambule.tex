%
% auteur : Maxime PINEAU
% version : 1.7
%
% mise à jour 1.7 : 
% - ajout des colorations syntaxiques de minted
% - %non-fini% tocbibind : ajout automatique des lof et lot dans le toc 
% - nomencl
%
% mise à jour 1.6 : 
% - atveryend : éviter erreurs etoolbox et pagelst (sharelatex)
%
% mise à jour 1.5 : 
% - un seul préambule 
% - etoolbox : if then else
% - titlesec, titletoc : changer style des sections
%
% mise à jour 1.4 : 
% - modification listing + ajout JavaScript (style sublimtext)
%
% mise à jours 1.3 : 
% - calc : calcule avec compteur latex (aurotise +, ...)
% - dcolumn : aligner colonne sur la décimale
% - siunitx : unité et format des nombres
% - auteur hyperref
% - changer color en xcolor
%


% Marges de la page
\usepackage[top=0.70in, bottom=0.70in, left=0.8in,right=0.80in, headheight=27.06pt]{geometry} % 'headheight' augmenté afin d'éviter warnings



% ********************************************************************
% Internationnalisation
% ********************************************************************
\usepackage[french, english, american]{babel}
\frenchbsetup{StandardLists=true} 	% evite erreurs de personalisation listes
\usepackage[autolanguage]{numprint}	% écrire nombre selon langue (\numprint)
% "numprint" obligatoirement à mettre après babel
\usepackage[detect-all]{siunitx} % unités et format des nombres (km-1, 1/km)



% ********************************************************************
% Accents
% ********************************************************************
\usepackage[T1]{fontenc}	% "font" utilisé pour afficher les caractères
\usepackage[utf8]{inputenc} % permet d'écrire les accents au clavier
% option de "inputenc" doit correspondre au format de sauvegarde du fichier 



% ********************************************************************
% Tout ce qui est textuel et mathématique
% ********************************************************************
\usepackage{pslatex} 	% times new roman, old package, but works
\usepackage{setspace}	% pour modifier l'espace entre les lignes
\usepackage{lipsum}
\usepackage{multicol} 	% séparer le texte sur plusieurs colonnes
\usepackage{amsmath,amsfonts,amssymb}	% Extensions pour les mathématiques
% pour changer le style des chapters, sections, ...
\usepackage[pagestyles]{titlesec}
% pagestyles : \usepackage{titleps}
\usepackage{titletoc}



% ********************************************************************
% Couleurs
% ********************************************************************
\usepackage[usenames,dvipsnames,svgnames,table]{xcolor}
% pour avoir les noms : http://latexcolor.com/ 
\definecolor{mauve}{rgb}{0.58,0,0.82}
\definecolor{cornsilk}{rgb}{1.0, 0.97, 0.86}
\definecolor{sublim background}{RGB}{39, 40, 34}
\definecolor{sublim string}{RGB}{230, 219, 116}
\definecolor{sublim keyword}{RGB}{249, 38, 89}
\definecolor{sublim comment}{RGB}{117, 113, 94}
\definecolor{sublim nbword}{RGB}{102, 217, 239}
\definecolor{sublim val}{RGB}{174, 129, 255}
\definecolor{monokai background}{HTML}{272822}



% ********************************************************************
% 
% ********************************************************************
\usepackage{pdfpages}	% pour incorporer un fichier PDF
\usepackage{url}		% comme une balise <a>
%\usepackage{import}	% importer des fichiers de façon plus précise ( \import{dossier le contenant}{nom fichier} )
\usepackage{comment}	% commentaires multi-lignes ( \begin{comment} )
\usepackage{calc}		% permet les calcules avec les compteurs latex
%\usepackage[nottoc]{tocbibind} 	% ajoute lot et lof dans la toc 

\usepackage{atveryend}  % chargé pour éviter erreurs entre pagelst et etoolbox
\usepackage{etoolbox}	% if then else et autres




% ********************************************************************
% Mode paysage
% ********************************************************************
\usepackage{pdflscape}	% page mode paysage \begin{landscape}

% variables temporaires pour mylandscape
\newlength{\newLeftMargin}
\newlength{\newTopMargin}
% mon landscape : \begin{mylandscape}....
\newenvironment{mylandscape} {
	\cleardoublepage 
	\begingroup
	% "tourner la page"
	\newpage
	\paperwidth=\pdfpageheight
	\paperheight=\pdfpagewidth
	\pdfpageheight=\paperheight
	\pdfpagewidth=\paperwidth
	% calcul des nouvelles marges
	\setlength{\newLeftMargin}{(\paperwidth - \textheight) / 2}
	\setlength{\newTopMargin}{(\paperheight - \textwidth) / 2}
	% modification des marges et de la zone du contenu (textheight...) 
	\newgeometry{textwidth=\textheight, textheight=\textwidth, left=\newLeftMargin, top=\newTopMargin}
	% ajuster la longueur du header selon la longueur de la page
	\headwidth=\textwidth
	% modifier la zone des paragraphes
	\vsize=\textheight
	\hsize=\textwidth
} {	
	\cleardoublepage
	\restoregeometry
	\endgroup
	% bien replacer le footer de la page suivante seulement
	% mauvaise solution...
	\begin{landscape}
	\end{landscape}
}



% ********************************************************************
% Listes
% ********************************************************************
\usepackage{paralist}	% une liste ayant ses éléments en ligne
\usepackage{enumerate}	% changer labels des énumérateurs (listes, ...)
\usepackage{enumitem}	% personaliser les listes 

% on essaie d'imiter la présentation de Babel
% voir : http://daniel.flipo.free.fr/frenchb/frenchb-doc.pdf
\newlength\mylabelwidth
\newcommand*{\mylabel}{\textemdash}  % ou \textendash (tiret plus court)
\settowidth{\mylabelwidth}{\mylabel}
\setlist[1]{labelindent=\parindent}
\setlist[itemize]{
	label=\mylabel, nosep, 
	labelwidth=\mylabelwidth,
	itemsep=0.4ex plus 0.2ex minus 0.2ex,
	parsep=0.4ex plus 0.2ex minus 0.2ex,
	topsep=0.8ex plus 0.4ex minus 0.4ex,
	partopsep=0.4ex plus 0.2ex minus 0.2ex,
	leftmargin=!
}
\setlist{ 
	labelwidth=\mylabelwidth, 
	itemsep=0.4ex plus 0.2ex minus 0.2ex,
	parsep=0.4ex plus 0.2ex minus 0.2ex,
	topsep=0.8ex plus 0.4ex minus 0.4ex,
	partopsep=0.4ex plus 0.2ex minus 0.2ex,
	leftmargin=!
}
%\setlist[itemize]{label={---}, leftmargin=45pt, parsep=0cm, itemsep=0cm, topsep=0cm} % ancienne personnalisation



% ********************************************************************
% Figures
% ********************************************************************
\usepackage{graphicx}	% pour inclure les images
\usepackage{float} 		% placement H pour les figures
\usepackage{subcaption}	% placer plusieurs figures cote à cote 
\usepackage{wrapfig}	% faire flotter une figure dans un paragraphe
\usepackage{grffile}	% autorise les " " dans les noms 
%\graphicspath{ {./images/} }   % répertoire des images par défaut
\usepackage[font=small,labelfont=bf]{caption}



% ********************************************************************
% Tableaux
% ********************************************************************
\usepackage{array} 		% texte en gras, modifier taille du tableau
\usepackage{tabularx}	% x : taille de colonne se calcule toute seule
\usepackage{dcolumn} 	% D{sep.tex}{sep.pdf}{maxDecimal} : aligner décimale
\usepackage{longtable}	% tableau de plus d'une page
\usepackage{multirow,makecell}	% fusionner cellules + autres (\thead ...)

% prochaines lignes pour palier à certains défauts des tableaux standards 
\setcellgapes{1pt}  	
\makegapedcells
\newcolumntype{R}[1]{>{\raggedleft\arraybackslash }b{#1}}
\newcolumntype{L}[1]{>{\raggedright\arraybackslash }b{#1}}
\newcolumntype{C}[1]{>{\centering\arraybackslash }b{#1}}
\newcolumntype{M}[1]{>{\raggedright}m{#1}}



% ********************************************************************
% Notes de bas de pages
% ********************************************************************
\usepackage{footnote} 			% \begin{savenotes} pour dans tableaux
\usepackage[bottom]{footmisc}	% footnote toujours collé en bas de page
\usepackage{fnpct}				% multiple footnotes 1, 2, 3, et autre : \multfootnote{;;}
%\setfnpct{dont-mess-around} % uncomment if don't want the kerning and punctuation switching



% ********************************************************************
% Indexe
% ********************************************************************
\usepackage[xindy]{imakeidx} 	% (plusieurs indexes possibles)
%\makeindex						% générer les indexes
%% \printindex <= a mettre quelque par pour afficher les indexes
%% utiliser "mot \index{mot}"



% ********************************************************************
% Nomenclature
% ********************************************************************
\usepackage[french, intoc]{nomencl}
%\makenomenclature
%% \printnomenclature
%% \nomenclature{$h$}{Planck constant}



% ********************************************************************
% Bibliographie et citations
% ********************************************************************
\usepackage[hyperref=true, backref=true, backend=biber, sorting=none, style=apa]{biblatex}
\DeclareLanguageMapping{american}{american-apa}
% sorting=none : pour ranger par ordre d'apparition
% http://www.latex-tutorial.com/tutorials/beginners/lesson-7/
\usepackage[babel=true]{csquotes} % pour la gestion des guillemets français.



% ********************************************************************
% Renommage
% - http://tex.stackexchange.com/questions/82993/how-to-change-the-name-of-document-elements-like-figure-contents-bibliogr
% ********************************************************************

% - sans Babel 
%\renewcommand{\figurename}{Fig.}
%\renewcommand{\contentsname}{Table of Contents}

% - avec Babel 
%\addto\captionsenglish{ 
%  \renewcommand{\figurename}{Fig.} 
%  \renewcommand{\contentsname}{Table of Contents} 
%}

% ! si utilisation de biblatex, le renommage doit se faire dans biblatex
% (\bibname et \refname ne marcheront pas)




% ********************************************************************
% Listing de code source
% ********************************************************************

%%%%%%%%%%%%%
% Linstings
%%%%%%%%%%%%%
\usepackage{listings} 

% pour ajouter d'autres "literates" en plus de ceux déjà existant (sans les effacer)
% attention, ne prend pas l'étoile '*'
% voir : http://tex.stackexchange.com/questions/149056/how-can-i-define-additional-literate-replacements-without-deleting-existing-ones
\makeatletter
\def\addToLiterate#1{\edef\lst@literate{\unexpanded\expandafter{\lst@literate}\unexpanded{#1}}}
\lst@Key{moreliterate}{}{\addToLiterate{#1}}
\makeatother

% utilisé pour bloquer la commande \textcolor
\newcommand{\textcolordummy}[2]{#2}


% pour les accents
\lstset{literate=
  {á}{{\'a}}1 {é}{{\'e}}1 {í}{{\'i}}1 {ó}{{\'o}}1 {ú}{{\'u}}1
  {Á}{{\'A}}1 {É}{{\'E}}1 {Í}{{\'I}}1 {Ó}{{\'O}}1 {Ú}{{\'U}}1
  {à}{{\`a}}1 {è}{{\`e}}1 {ì}{{\`i}}1 {ò}{{\`o}}1 {ù}{{\`u}}1
  {À}{{\`A}}1 {È}{{\'E}}1 {Ì}{{\`I}}1 {Ò}{{\`O}}1 {Ù}{{\`U}}1
  {ä}{{\"a}}1 {ë}{{\"e}}1 {ï}{{\"i}}1 {ö}{{\"o}}1 {ü}{{\"u}}1
  {Ä}{{\"A}}1 {Ë}{{\"E}}1 {Ï}{{\"I}}1 {Ö}{{\"O}}1 {Ü}{{\"U}}1
  {â}{{\^a}}1 {ê}{{\^e}}1 {î}{{\^i}}1 {ô}{{\^o}}1 {û}{{\^u}}1
  {Â}{{\^A}}1 {Ê}{{\^E}}1 {Î}{{\^I}}1 {Ô}{{\^O}}1 {Û}{{\^U}}1
  {œ}{{\oe}}1 {Œ}{{\OE}}1 {æ}{{\ae}}1 {Æ}{{\AE}}1 {ß}{{\ss}}1
  {ű}{{\H{u}}}1 {Ű}{{\H{U}}}1 {ő}{{\H{o}}}1 {Ő}{{\H{O}}}1
  {ç}{{\c c}}1 {Ç}{{\c C}}1 {ø}{{\o}}1 {å}{{\r a}}1 {Å}{{\r A}}1
  {€}{{\EUR}}1 {£}{{\pounds}}1
}

\lstset{ %
	language=Java,			% langage par défaut 
	escapeinside={(*@}{@*)},% pour ajouter du Latex dans le code
	%morekeywords={*,...},	% pour ajouter plus de mots-clefs
	%otherkeywords={>,<,.,;,-,!,=,~,+,*,&,|},
    %morekeywords={>,<,.,;,-,!,=,~,+,*,&,|},
	%keepspaces=true,		% garder l'indentation du code, keep space (possibly needs columns=flexible)	
	% ------- Style ------ 
	backgroundcolor=\color{sublim background},
	basicstyle=\color{white}\normalsize,
	keywordstyle=\color{sublim keyword}, %\usefont{OT1}{cmtt}{m}{n},
	ndkeywordstyle=\bfseries\color{sublim nbword},
	identifierstyle=\ttfamily\color{white},
	commentstyle={\slshape\color{sublim comment}\let\textcolor\textcolordummy}, %\color{purple}\ttfamily, 
	stringstyle={\slshape\color{sublim string}\let\textcolor\textcolordummy}, %\color{red}\ttfamily,
	% ------- Contour du listing -------
	title=\lstname,	% affiche nom des fichiers lors de \lstinputlisting;
	captionpos=b,	% position du caption en bottom		
	frame=trBL,	% fenêtre autour du code (single, trBL, ...)
	framerule=1pt,	% taille de la bordure 
	framesep=4pt,	% padding entre bordure et contenu 
	rulesep=1pt,	% espacement entre les deux bordures si trBL (?)
	rulecolor=\color{black},% if not set, the frame-color may be changed on line-breaks within not-black text
 	% ------- Numérotation ligne -------
  	numbers=left,	% position des numéros de ligne
	stepnumber=1,	% chaque ligne est numérotée
	numbersep=20pt,	% espacement entre numéro ligne et le code (5pt)
	numberstyle=\tiny\color{gray},  % style des numéros de lignes
	% ------- Indentation et Saut de ligne -------
	breaklines=true,		% saut de ligne automatique
	breakatwhitespace=false,% saut de ligne seulement lors d'un espace 
	tabsize=2,				% tabulation de 2 espaces 
	showspaces=false,		% afficher les espaces avec des underscores 
	showstringspaces=false,	% souligner les espaces dans une chaine 
	showtabs=false,			% afficher tabulation dans les chaines avec des underscores  		
	% ------- Style des nombres et opérateurs ------ 
	moreliterate=
		% opérateurs
		{>}{{\textcolor{sublim keyword}{>}}}{1}
		{<}{{\textcolor{sublim keyword}{<}}}{1}
		{.}{{\textcolor{sublim keyword}{.}}}{1}
		{;}{{\textcolor{sublim keyword}{;}}}{1}
		{-}{{\textcolor{sublim keyword}{-}}}{1}
		{+}{{\textcolor{sublim keyword}{+}}}{1}
		{!}{{\textcolor{sublim keyword}{!}}}{1}
		{=}{{\textcolor{sublim keyword}{=}}}{1}
		{~}{{\textcolor{sublim keyword}{~}}}{1}
		{&}{{\textcolor{sublim keyword}{&}}}{1}
		{|}{{\textcolor{sublim keyword}{|}}}{1}
		% nombres
		{0}{{\textcolor{sublim val}{0}}}{1}
		{1}{{\textcolor{sublim val}{1}}}{1}
		{2}{{\textcolor{sublim val}{2}}}{1}
		{3}{{\textcolor{sublim val}{3}}}{1}
		{4}{{\textcolor{sublim val}{4}}}{1}
		{5}{{\textcolor{sublim val}{5}}}{1}
		{6}{{\textcolor{sublim val}{6}}}{1}
		{7}{{\textcolor{sublim val}{7}}}{1}
		{8}{{\textcolor{sublim val}{8}}}{1}
		{9}{{\textcolor{sublim val}{9}}}{1}
		% nombres à virgules
		{.0}{{\textcolor{sublim val}{.0}}}{2}
		{.1}{{\textcolor{sublim val}{.1}}}{2}
		{.2}{{\textcolor{sublim val}{.2}}}{2}
		{.3}{{\textcolor{sublim val}{.3}}}{2}
		{.4}{{\textcolor{sublim val}{.4}}}{2}
		{.5}{{\textcolor{sublim val}{.5}}}{2}
		{.6}{{\textcolor{sublim val}{.6}}}{2}
		{.7}{{\textcolor{sublim val}{.7}}}{2}
		{.8}{{\textcolor{sublim val}{.8}}}{2}
		{.9}{{\textcolor{sublim val}{.9}}}{2},
}
% pour l'ajout des nombres, voir : http://tex.stackexchange.com/questions/40284/how-to-color-digits-with-the-listings-package
% on utilise '\let\textcolor\textcolordummy' pour ne pas afficher la couleur des nombres et opérateurs


% ajout du langage JavaScript
\lstdefinelanguage{JavaScript}{
	% ------- Mots clés ------
	morekeywords={js, pow, math.pow, break, const, continue, delete, do, while, export, for, in, if, else, typeof, instanceof, label, let, new, return, switch, case, throw, try, catch, implements, with, yield, import},
	% ------- Autres Motes clés ------	
	ndkeywords={Object, Array, Math, Boolean, boolean, Number, Function, function, class, void, this, var}, 
	% ------- Commentaire ------
	comment=[l]{//},
	morecomment=[s]{/*}{*/},
	% ------- String -------
	morestring=[b]',
	morestring=[b]",
	% ------- Valeurs en plus ------
    classoffset=3,	% starting a new class
    morekeywords={true, false, undefined, null, Infinity},
    keywordstyle=\color{sublim val},
    classoffset=0,	% restore to default class if more customisations...
	% ------- Style ------
	backgroundcolor=\color{sublim background},
	basicstyle=\color{white}\normalsize,
	keywordstyle=\color{sublim keyword}\usefont{OT1}{cmtt}{m}{n},
	ndkeywordstyle=\bfseries\color{sublim nbword},
	identifierstyle=\ttfamily\color{white},
	commentstyle={\slshape\color{sublim comment}\let\textcolor\textcolordummy}, %\color{purple}\ttfamily, 
	stringstyle={\slshape\color{sublim string}\let\textcolor\textcolordummy}, %\color{red}\ttfamily,
	% ------- Autre ------
	sensitive=false,
	keepspaces=false
}

% \lstlistoflistings % pour afficher la liste des listings



%%%%%%%%%%%%%
% Minted 
% - a besoin de : Python 2.6 (ou plus) et Pygments ("pip install Pygments")
% - exécuter avec : 
% latex --shell-escape <file>
% pdflatex --shell-escape  <file>
% - https://fr.sharelatex.com/learn/Code_Highlighting_with_minted 
% - http://distrib-coffee.ipsl.jussieu.fr/pub/mirrors/ctan/macros/latex/contrib/minted/minted.pdf
%%%%%%%%%%%%%
%\usepackage[chapter]{minted}
\usepackage{minted}
% chapter : pour afficher numéro listing sous la forme Chapter.number


% renommage 
\renewcommand{\listoflistingscaption}{Liste des listings}	%\listoflistings
%\renewcommand{\listingscaption}{Program code}

% autres utilisation : 
% - \newminted{langue}{options} pour définir "XXXcode" environnement
% - \inputminted[options]{language}{cheminFichier}



%%%
% Paramétrage minted
%%%
\setminted{
	% ------- Numéro lignes ------
	linenos, 			% numéroter les lignes
	% numbersep = 20pt, % (12pt) espace entre les numéro et les lignes de code
	% numbers = left, 	% position numéro de ligne (left, right)
	% ------- Breaklines ------	
	breaklines, 		% autorise de casser une ligne si nécessaire (symbole \breakaftersymbolpre)
	breakautoindent, 	% réindenter une ligne cassée 
	breakindent = 2pt,	% indentation ajouté aux lignes cassées 
	% ------- Font ------	
	% fontfamily = ..
	fontsize = \footnotesize,
	% ------- Style ------ 
	% style = monokai, 
	% bgcolor = monokai background,
	% ------- Frame ------	
	frame = lines,		% none, leftline, topline, bottomline, lines, single
	framerule = 0.01pt, 	% largeur de la frame 0.4pt
	% framesep = 6pt, 		% distance entre frame et contenu (\fboxsep)
	% rulecolor = \color{sublim comment}, 	% couleur de la frame 
	% xleftmargin = 0, 		% marge à ajouter à gauche du listing 
	% xrightmargin = 0, 	% marge à ajouter à droite du listing 
	% ------- Autre ------
	mathescape, % autorise l'utilisation des formules mathématiques
	autogobble, % retire les espaces inutiles et indente 
	escapeinside =@@,
}


\setminted[java]{
	codetagify = [@Override],	% ajout d'un mot clé 
}



% styles utilisés
\usemintedstyle[bash]{bw}
\usemintedstyle[bat]{bw}

\usemintedstyle[java]{borland}
\usemintedstyle[python]{emacs}

\usemintedstyle[c]{vs}
\usemintedstyle[cpp]{vs}
\usemintedstyle[csharp]{vs}

\usemintedstyle[haskell]{default}
\usemintedstyle[ocaml]{murphy}
\usemintedstyle[prolog]{manni}

\usemintedstyle[sql]{borland}
\usemintedstyle[mysql]{borland}

\usemintedstyle[tex]{tango}
\usemintedstyle[matlab]{perldoc}

\usemintedstyle[php]{pastie}
\usemintedstyle[html]{default}
\usemintedstyle[hxtml]{default}
\usemintedstyle[xml]{default}

\setminted[js]{
	style = monokai,
	bgcolor = monokai background,
}


% http://tex.stackexchange.com/questions/269491/mixing-minted-with-lstlisting 


%%%
% Paramétrage tcolorbox
% - background color
% - frame 
%%%
%\usepackage[most]{tcolorbox}	% pour background et frame 
%\tcbuselibrary{minted}
%\tcbset{
%	listing engine = minted,
%	minted options = {
%		numbersep = 4mm,
%	},
%	left = 1mm,
%	listing only,
%	breakable,
%}
%\newtcblisting{java}[1][]{
%	minted language = java,
%	colback = blue!5!white,
%	colframe = blue!75!black,
%	#1,
%	%minted options = {fontsize=\small,numbersep=3mm},
%}
%\newtcbinputlisting{\javafile}[2][]{
%	minted language = java,
%	colback = red!5!white,
%	colframe = red!75!black,
%	listing file = {#2},
%	#1
%}



% ********************************************************************
% Hyperreferences
%
% - package à mettre après hyperref :  		http://texblog.net/hyperref/
% cleveref, amsrefs, algorithm, chappg, sidecap, minitoc, linguex
% robustindex, hypdestopt, hypcap, hypbmsec, hypernat, attachfile
% uri, pageslts, regstats, refcheck, bookmark, glossaries, apacite
% geometry (avec mag <> 1000), footnotebackref
% ********************************************************************
\usepackage[%dvips, % commented for pdflatex
bookmarks, colorlinks=false]{hyperref} % liens dans le PDF
% hyperfootnotes=false 
\hypersetup{%
	pdfborder = {0 0 0},
	pdfauthor = {Maxime PINEAU}
	%pdftitle={},
}

% packages après hyperref
\usepackage[french]{cleveref}	% \cref{...} : eq. 1, fig. 1, tab. 1
\usepackage[symbol=${}^{\scriptscriptstyle\uparrow}\,\,\,$]{footnotebackref} % ne marche pas bien avec Babel
% $\mkern-3\thinmuskip{}^{\scriptscriptstyle\uparrow}\,\,\,$

% résoudre les problèmes d'indentation dans toc (\bookmarksetup{startatroot})
% - https://groups.google.com/forum/#!topic/comp.text.tex/LBr2vChK-QE
%\usepackage{bookmark} 



% ********************************************************************
% Glossaire
% ********************************************************************
\usepackage[footnote]{glossaries}	% footnote (mots dans le footer)
% toc, section=part => attention, va ajouter une page de partie
\makeglossaries						% permet de générer le glossaire



% ********************************************************************
% En-têtes et Pieds de page
% ********************************************************************
\usepackage{fancyhdr}	% ajoute les styles pour les header / footer
\usepackage{pageslts}	% pour compter le nombre de pages

\pagestyle{fancy}		% utiliser style "fancy" (header/footer)

%%%
% Style fancy
% - style des pages normales
%%%
\lhead{} 	% \lhead{\nouppercase{\rightmark}}
\cfoot{}	% ou : \fancyfoot[C]{}
\lfoot{\emph{École Polytechnique de l'Université de Nantes, Département Informatique}}
\rfoot{ Page ~\thepage ~ / ~\lastpageref{pagesLTS.arabic} }
\renewcommand{\headrulewidth}{0.4pt}	% trait horizontal en-tête
\renewcommand{\footrulewidth}{0.4pt}	% trait horizontal pied de page

% on modifie la façon d'afficher le titre du chapitre
%\renewcommand{\chaptermark}[1]{\markboth{\MakeUppercase{\ #1}}{}} 


%%%
% Style plain
% - utilisé lors d'un début de chapitre
%%%
\fancypagestyle{plain}{	
	\lhead{}
	\rhead{}	
	\lfoot{\emph{École Polytechnique de l'Université de Nantes, Département Informatique}} 
	\rfoot{ Page ~\thepage ~ / ~\lastpageref{pagesLTS.arabic} }	
	\renewcommand{\headrulewidth}{0pt}
	\renewcommand{\footrulewidth}{0.4pt}
	% header : rien mettre pour reprendre celui de "fancy"
	% normalement, pas besoin de tout redéfinir...	
}

%%%
% Style annexes
%%%
\fancypagestyle{appendix}{
	% affiche le compteur en 1, 2, 3
	\renewcommand*{\thepage}{.\arabic{page}}	
	
    % modifie le footer pour avoir le bon compteur de pages (le Romain)
    \fancyfoot[R]{ Page ~\thepage ~ / ~\lastpageref{pagesLTS.Roman} }

	% on modifie le style plain pour avoir le bon compteur
	\fancypagestyle{plain}{
		\rhead{}
		\renewcommand{\headrulewidth}{0pt}
		\fancyfoot[R]{Page ~\thepage ~ / ~\lastpageref{pagesLTS.Roman}}
	}

}




% ********************************************************************
% Autre
% ********************************************************************
\overfullrule=2cm	% met une barre noire lors d'un warning "Overfull \hbox"

% \usepackage{layout} ... \layout{}
