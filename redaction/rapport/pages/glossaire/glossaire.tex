% pour ajouter le glossaire dans la table des matières
\cleardoublepage	% évite les erreurs de numérotation
\phantomsection		% évite les erreurs de placement de lien



%\glstoctrue		


%\addcontentsline{toc}{part}{Glossaire}
%\printglossary[title=Glossaire, toctitle=Glossaire]
\printglossary
% title : le nom de la page qui apparait à gauche en début de page
% toctitle : apparait en haut à droite dans le header


% pour les explications :
% https://en.wikibooks.org/wiki/LaTeX/Glossary




% \newglossaryentry{<label>}{
%	 name=<nom>,
%	 description={<description>},
%	 fisrt={<premier appel>},
%	 plural={<si on ajoute pas de 's' à la fin>},
%	 symbol={<symbole pi>},
% 	 see={<label_a_voir}
% }

% \newacronym{<label>}{<abbreviation>}{<full>}
% \newacronym{<label>}[longplural={pluriel}]{<abbreviation>}{<full>}


% utilisation (pour tout):
% - \gls(<label>)
% - \Gls(<label>)  		=> avec majuscule
% - \glspl(<label>)  	=> pluriel
% - \Glspl(<label>)		=> pluriel avec majuscule
% - \glsdesc{<label>}  	=> affiche la description
% - \glssymbol{<label>}	=> affiche le symbole

% utilisation en plus pour les accronymes : 
% - \acrshort{<label>}	=> donne l'abbréviation
% - \acrlong{<label>}	=> donne la description "full"
% - \acrfull{<label>}	=> donne "full (abbreviation)"




