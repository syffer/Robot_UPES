\chapter[Presentation of the UPES]{Presentation of the University of Petroleum and Energy Studies}

The objective of this chapter is to present the University of Petroleum ans Energy Studies, the IT department and the differents persosn working on the project I was affected.

\section{The school}


\begin{wrapfigure}{r}{0.5\textwidth}
	\centering
	\includegraphics[width=0.4\textwidth]{images/upes/location}	
	\caption{UPES Geographic localisation \cite{bib:upes:map}}
	\label{fig:upes:localisation}
\end{wrapfigure}

The \gls{UPES} is an indian private university school. It was created in 2003. The university is located in Dehradun, the capital city of the state Uttarakhand. The Uttarakhand is located in the northern part of India, near the Himalayas.  

~~

The \gls{UPES} offers industry-focused specialized graduate, post graduate and doctoral programs. The university's  multidisciplinary and specialized courses are related to fourteen  different key sectors, such as transportation, power, gas, computer science, information technology, logistics, infrastructur, design, and legal studies. It provides over 87 undergraduate and postgraduate programs. It has three colleges, four schools and a center in the campus. The university has also twelve \gls{RandD} centers. 

~~

It is the first the first indian energy institution to set up an in-house bio-diesel laboratory. The university was also the first indian energy university to be recognized by the \gls{UGC}, a statutory body set up by the Indian Union government, and charged of the coordination, determination and maintenance of the standards of higher education, it provides recognition to universities in India. The \gls{UPES} is also the first indian university under public-private partnership, and has an incubator who interacts with medium-small scale entreprises and the Government of Uttarakhand. 

~~

The university posess two campuses which are separated by 3 km from each others, the Bidholi and the Kandoli campuses (both in Derhadun). The Bidholi campus is composed of nine blocks (or buildings), the \emph{Energy House} is considered as the main building. The university also possess a working drilling rig behind his \gls{RandD} centers, dedicated for the students \cite{bib:upes:wikipedia}.

~~ 

It became part of the \emph{Laureate International Universities} network in 2013, which include over eighty accredited universities in twenty-five differents countries. \emph{Laureate International Universities} are univercities and colleges owned and operated by the company \emph{Laureate Education}, it is an international community that spans the America, Europe, Africa Asia and the Middle East \cite{bib:laureate:site}. The \gls{UPES} has made some partnership with differents universities over the world, including french universities of  Polytech network and the \gls{ENSEM}. During summer 2016, sixteen french students were present at the Bidholi campus. 


\section{IT department and my Work team}

I was affected at the \gls{IT} department of the \gls{UPES}, which is under the direction of Dr. Manish Prateek. Dr. Manish Prateek is a professor and an associate dean at the \gls{UPES}, he has a PhD in robotics. 

~~

The principal research subjects of the \gls{IT} department are about signal and \gls{image processing}, robotics and artificial intelligence. The \gls{IT} department has also multiple ongoing projects involving differents areas. For example in the \gls{image processing}, there are projects about the recognition and the identification of tigers through his stripes using an image. 

~~

I was placed under the direction of Ms. Niharika SINGH, an assistant professor, specialized in information technology, artificial intelligence and artificial neural networks. She has a master's degree in those previous specializations and a engineer's degree in electronics and communication.

~~

%During my internship, I mostly 