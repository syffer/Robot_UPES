\chapter{Presentation of the University of Petroleum and Energy Studies}


\section{The school}


\begin{wrapfigure}{r}{0.5\textwidth}
	\centering
	\includegraphics[width=0.4\textwidth]{images/upes/location}	
	\caption{UPES Geographic localisation \cite{bib:upes:map}}
	\label{fig:upes:localisation}
\end{wrapfigure}

The \acrfull{UPES} is an indian private university school. It was created in 2003. The university is located in Dehradun, the capital city of the state Uttarakhand. The Uttarakhand is located in the northern part of India, near the Himalayas.  

~~

The \acrshort{UPES} offers industry-focused specialized graduate, post graduate and doctoral programs. The university's  multidisciplinary and specialized courses are related to fourteen  different key sectors, such as transportation, power, gas, computer science, information technology, logistics, infrastructur, design, and legal studies. It provides over 87 undergraduate and postgraduate programs. It has three colleges, four schools and a center in the campus. The university has also twelve \acrfull{RandD} centers. 

~~

It was the first the first indian energy institution to set up an in-house bio-diesel laboratory. The university was also the first indian energy university to be recognized by the \acrfull{UGC} (a statutory body set up by the Indian Union government, and charged of the coordination, determination and maintenance of the standards of higher education, it provides recognition to universities in India). The \acrshort{UPES} is also the first indian university under public-private partnership, and has an incubator who interacts with medium-small scale entreprises and the Government of Uttarakhand. 

~~

The university posess two campuses which are separated by 3 km from each others, the Bidholi and the Kandoli campuses (both in Derhadun). The Bidholi campus is composed of nine blocks (or buildings), the \emph{Energy House} is considered as the main building. The university also possess a working drilling rig behind his \acrshort{RandD} centers, dedicated for the students \cite{bib:upes:wikipedia}.

~~ 



\section{IT department and my Work team}


