
\thispagestyle{empty}

\vfill

\begin{center}
	\LARGE{\textbf{RÉSUME}}\\[1.0cm]
\end{center}

% 1/2 page chacun

%resumé en français
\large{ 
Ce nouveau project concerne la conception d'un algorithm de navigation automatique utilisant l'informatique symbolique. Ce rapport ne concerne qu'une petite partie de ce projet, qui a un rapport avec l'implémentation de différents transformations de traitement d'image, et une reconnaissance d'objet sur une image 2D. Une interface graphique est fournie afin qu'un utilisateur quelconque puisse charger une image et tester chacune des transformations implémentées. Des transformations de base sont disponibles, comme par exemple passer une image en échelle de gris, appliquer un seuillage, obtenir le négatif d'une image, et d'autres opérations morphologiques. Les opérations de détection de contours implémentées sont les suivants : le s filtres de sobel, canny et laplacian, ainsi qu'une version 2D de l'algorithm CLAP. Des opérations de réduction de bruit sont également disponible, comme par exemple les filtres médian et moyenneur pondéré, ainsi qu'un algorithme utilisant un automate cellulaire. La reconnaissance d'object utilise les caractéristiques des chain codes de Freeman, après avoir appliqué une détection de contour et un seuillage. Une petite partie du jeu de donnée mis à disposition par le projet LabelMe est utilisé afin de connaître quelques objets déjà étiquetés, et une base de donnée sous Oracle est utilisé afin de les stocker.
}

~~

\textbf{Mots-clés: } traitement d'image, automate cellulaire, CLAP, détection de contour, réduction du bruit, freeman chain code, reconnaissance d'object


\vfill


\begin{center}
	\LARGE{\textbf{ABSTRACT}}\\[1.0cm]
\end{center}


%résumé en anglais %
\large{ 
This new project involves designing an autonomous navigation algorithm using symbolic computing. This report concerns a very little part of this project which regards an implementation of different image processing transformations, and an object recognition in a 2D image. A GUI is provided in order for a user to test each transformation. Basis transformations are available, like passing an image into a grey scale, applying a threshold, get its negative, and some morphological operations. Implemented edge detection operators are the sobel, the canny and the laplacian filters, along side with the 2D version of the CLAP algorithm. Some noise removal are also present, like the median filter, the weighted average filter, and a cellular automata algorithm. The object recognition uses the freeman chain code features afer applying an edge detection and a threshold transformation. A little part of the dataset provided by LabelMe project is used in order to know some tagged objects, and an Oracle database is employed in order to store them.
}

~~

\textbf{Keywords: } image processing, cellular automata, CLAP, edge detection, noise removal, freeman chain code, object recognition

\vfill
