\chapter{Other works}


\section{Set Theoretic Rajan Transform}

We planned to use the \gls{STRT} in our process but it wasn't possible due to some issues and imprecisions in the documentation concerning the \gls{STRT}$^{-1}$. We wanted to use this transformation on an image mainly because it has noise removal properties.

~~

The \gls{STRT} correspond to the \gls{RT} applied on the sets domain. First I will present the forward rajan transform, and then it's application in the sets domain. Every equations, figures and explainations have been inspired by the following reference  \cite{bib:symbolic:RajanTransform}.

~~

The rajan transform take a sequence of $2^{i}$ numbers (the number of elements of the sequence have to be a power of 2), transform it, and return another sequence of $2^{i}$ numbers. We will call $x(k)$ the input sequence, $X(k)$ the output sequence of the tranform, and $N$ the number of element in the sequence \cite{bib:symbolic:RajanTransform}. 

~~

We can define the sequences $x(k)$, $g(k)$ and $h(k)$ as following : 
\begin{align}
x &= x(0), x(2), \cdots, x(k-1) & \text{where k is a power of 2} \\
g(k) &= x(k) + x(k + \frac{N}{2}) & \text{with } 0 \leq k \leq N / 2 \\
h(k) &= | x(k) - x(k - \frac{N}{2}) | & \text{with } N / 2 \leq k \leq N
\end{align}

~~

In other words, the sequence $x$ will be divided in two. Then we will sum the two first value of the subsequences (which will give us the results of $g(1)$), then the two seconds values (which is $g(2)$), then the thirds values, and so on until all the elements were processed. Then we will do the same operations but with a substraction instead (to get the sequence $h$).

~~

This processus will then have to be repeated on the subsequences $g$ and $h$ separately. We can see here the recursive character of this transformation. The figure \ref{fig:diagram:flowchart:rajan} illustrates the rajan tranformation in a more procedural way, with a sequence of 4 elements. 

\begin{figure}[ht]
	\centering
	\includegraphics[width=0.7\textwidth]{images/diagrams/flowchart_rajan_transform}
	\caption{The forward Rajan Tranfform (RT)}
	\label{fig:diagram:flowchart:rajan}	
\end{figure}

~~

The \gls{STRT} is the application of the Rajan Transform in the sets domain, and so instead of tranforming a sequence of numbers, it will tranform a sequence of sets, and return another sequence of sets (e.g. $x(1)$ and $X(1)$ wouldn't be numbers, but two sets). In the set domain, the addition correspond to the union, and the substraction to the difference of two sets.



\section{Tagging algorithm pseudocode}


\begin{listing}[H]
	\begin{minted}[mathescape=true]{java}
public AnnotatedObjectBean getNearestAnnotated(SomeObject object, List<AnnotatedObjectBean> others) {
	pivot, lower @$\leftarrow$@ 0, difference @$\leftarrow$@ 0
	higher @$\leftarrow$@ others.size()
	
	while (lower < higher) { // get an estimation of the nearest object position 
		pivot @$\leftarrow$@ (lower + higher) / 2
		AnnotatedObjectBean other @$\leftarrow$@ others.get(pivot)
		difference @$\leftarrow$@ object.compareTo(other)
		
		if (difference > 0) lower @$\leftarrow$@ pivot + 1  // change the boundaries accordingly 
		else if (difference < 0) higher @$\leftarrow$@ pivot;
		else return other;
	}
	
	difference @$\leftarrow |$@difference@$|$@  // absolute value to get the nearest
	for (i in [pivot-1, pivot+1]) {
		if (i < 0 || i >= others.size()) continue
		
		AnnotatedObjectBean other @$\leftarrow$@ others.get(i)
		diff = @$|$@object.compareTo(other)@$|$@
		
		if (diff < difference) { // a nearest object has been found
			difference @$\leftarrow$@ diff
			pivot @$\leftarrow$@ i
		}
	}
	
	return others.get(pivot)	
}
	\end{minted}
	\caption{Simplified tagging algorithm pseudocode}
	\label{lst:simplified tagging algorithm}
\end{listing}




\section{The CLAP algorithm}
\label{sec:annexe:clap}

The \gls{CLAP} algorithm as given in the paper \cite{bib:filter:EdgeWithCLAP}, for finding edges in a 3D images : 
\begin{enumerate}
	\item read the 3-D data and place voxel values in a 1-D array called "input\_array"
	\item copy "input\_array" to "output\_array"
	\item repeat sliding the 7-neighborhood window over the image(input\_array) until the structuring element spans whole of image 
	\begin{enumerate}
		\item find the maximum gray value $G_{max}$
		\item find the minimal gray value $G_{min}$
		\item find the differende $D = G_{max} - G_{min}$
		\item if($D \leq $ threshold) then assign zero value to all seven cells in "output\_array" 
		\item slide the 7-neighborhood
	\end{enumerate}
	\item pass the "output\_array" to VolumeRendered() method 
\end{enumerate}

The adapted version for 2D images involves only a 5-neighborhood instead of a 7. 

