
\begin{titlepage}	% Pour réaliser une page de couverture

\begin{center}



% images en haut de page
\begin{minipage}[t]{0.48\textwidth}
	\begin{flushleft}
		\includegraphics [width=60mm]{images/logo_ecoles/Polytech_Nantes_Universite} \\
	\end{flushleft}
\end{minipage}
\begin{minipage}[t]{0.48\textwidth}
	\begin{flushright}
		\includegraphics [width=50mm]{images/logo_ecoles/Universite_de_Nantes_} \\
		%\textsc{\LARGE Entreprise}
	\end{flushright}
\end{minipage} 

\vfill

\LARGE{\textbf{RAPPORT}} 

\vfill

\Large{\textbf{Remis à}} \\
\LARGE{\textbf{L'École Polytechnique de l'Université de Nantes\\Département Informatique}} 

\vfill 


% commentaire multi-lignes
\begin{comment}				%%%%%%%%%%%%%%%%%%%%%%%%%%%%%%%%%%%%%%%%%%

% les personnes concernées
\begin{table}[h] % h pour 'here', à coté du texte
	\centering	% on centre le tableau sur la page
	\Large{
	\begin{tabular}{>{\bfseries}lc>{\bfseries}r}
		\multicolumn{3}{>{\bfseries}c}{Réalisé par}	\\
		Maxime && \textsc{Pineau}		\\ 	% saut de ligne avec '\\'
	\end{tabular}
	% on laisse une colonne vide au milieu pour faire un espace
	% >{\bfseries}l : met en gras et alignement à gauche
	% >{\bfseries}r : met en gras et alignement à droite
	}
\end{table} 				%%%%%%%%%%%%%%%%%%%%%%%%%%%%%%%%%%%%%%%%%%

\end{comment}


\Large{\textbf{Réalisé par}} \\
\Large{\textbf{Maxime PINEAU}} 

\vfill

\large{\textbf{\today}}

\vfill

%\large{\textbf{D\'{E}PARTEMENT INFORMATIQUE}}\\
%\Large{\textbf{INSTITUT UNIVERSITAIRE DE TECHNOLOGIE}}\\
%\large{\textbf{NANTES - FRANCE}}
%\large{\textbf{\\2013-2014}}\\[0.5cm]
%\includegraphics[scale=1.4]{images/logo_ecoles/Universite_de_Nantes_}\\


\end{center}

\end{titlepage}