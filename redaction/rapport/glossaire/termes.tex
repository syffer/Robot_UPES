

% \newglossaryentry{<label>}{
%	 name = {<nom>},
%	 description = {<description>},
%	 first = {<premier appel>},
%	 plural = {<si on ajoute pas de 's' à la fin>},
%	 symbol = {<symbole pi>},
% 	 see = {<label_a_voir},
% }

% \newacronym{<label>}{<abbreviation>}{<full>}
% \newacronym[longplural={pluriel}]{<label>}{<abbreviation>}{<full>}


% utilisation (pour tout):
% - \gls(<label>)
% - \Gls(<label>)  		=> avec majuscule
% - \glspl(<label>)  	=> pluriel
% - \Glspl(<label>)		=> pluriel avec majuscule
% - \glsdesc{<label>}  	=> affiche la description
% - \glssymbol{<label>}	=> affiche le symbole

% utilisation en plus pour les accronymes : 
% - \acrshort{<label>}	=> donne l'abbréviation
% - \acrlong{<label>}	=> donne la description "full"
% - \acrfull{<label>}	=> donne "full (abbreviation)"


\newglossaryentry{mot}{
	name = mot,
	description = {un mot},
	%first = {le premier mot (mot))}
}




\newglossaryentry{AI}{
	name = {AI},
	description = {Artificial Intelligence is the ability of a computer or other machine to perform those activities that are normally thought to require intelligence (intelligence displayed by machines)},
	first = {Artificial Intelligence (AI)},
}

\newacronym{API}{API}{Application Programming Interface}

\newglossaryentry{algorithm}{
	name = {algorithm},
	description = {a way of resolving a problem. It is a sequence of finit operations (or instruction) allowing to resolve a problem or obtain a result},
	see = {program},
}



\newacronym{BTech}{BTech}{Bachelor of Technology}



\newacronym[longplural={Cellular Automata}]{CA}{CA}{Cellular Automaton}

\newacronym{CLAP}{CLAP}{Cellular Logic Array Processing}


\newglossaryentry{classification}{
	name = {classification},
	description = {the problem of identifying to which of set of categories (sub-population, cluster) a new observation belongs. The goal is to found the best data partition},
	see = {individual,cluster,population},
}

\newglossaryentry{cluster}{
	name = {cluster},
	description = {a set of similar individuals. The individuals inside a cluster are more similar to each other than to individials in other clusters},
	see = {individual,classification},
}

\newglossaryentry{cluster analysis}{
	name = {cluster analysis},
	description = {the task of grouping a set of individuals by their similarities},
	see = {cluster},
}


\newglossaryentry{data analysis}{
	name = {data analysis},
	description = {a process applied on a set of observed data, which is called the population. The goal is to study the dataset in a global manner, and discover useful information},
	see = {individual,population},
}


\newacronym{DAO}{DAO}{Data Access Object}

\newacronym{DB}{DB}{database}

\newacronym{DBMS}{DBMS}{Database Management System}

\newacronym{DIP}{DIP}{Digital Image Processing}


\newacronym{EM}{EM}{Expectation-Maximization algorithm}

\newglossaryentry{ENSEM}{
	name = ENSEM,
	description = {Ecole Nationale Supérieure d'Electricité et de Mécanique (National Higher School of Electricity and Mecanics)},
	first = {ENSEM (National Higher School of Electricity and Mecanics)},
}





\newacronym{GUI}{GUI}{Graphic User Interface}



\newglossaryentry{histogram}{
	name = {histogram},
	description = {a graphical representation of the distribution of numerical data (the number of appearences of each numerical data). For an image, those numerical data represents pixels values},
}



\newglossaryentry{image processing}{
	name = {image processing},
	description = {In image science, it is the processing of images using mathematocal operations, by using any form of signal processing and where the input could be an image or a video},
}

\newglossaryentry{individual}{
	name = {individual},
	description = {also called an observation, it is described by a set of variables (i.e. values)},
	see = {data analysis,classification,population},
}

\newacronym{IT}{IT}{Information Technologie}



\newacronym{MTech}{MTech}{Master of Technology}

\newacronym{MVC}{MVC}{Model View Controller}


\newacronym{PSO}{PSO}{Particle Swarm Optimization}

\newglossaryentry{population}{
	name = {population},
	description = {a set of individuals that where observed and collected through diverse sources},
	see = {individual,classification,data analysis},
}

\newglossaryentry{process}{
	name = {process},
	description = {an execution of a program},
	see = {program},
}

\newglossaryentry{program}{
	name = {program},
	description = {the implementation of an algorithm in a programming language},
	plural = {processes},
	see = {algorithm},
}



\newacronym{UGC}{UGC}{University Grants Commission}

\newacronym{UPES}{UPES}{University of Petroleum and Energy Studies}






\newacronym{RandD}{R\&D}{Research and Development}

\newacronym[see={API}]{SAX}{SAX}{Simple API for XML}

\newacronym{STRT}{STRT}{Set Theoretic Rajan Transform}
\newacronym{RT}{RT}{Rajan Transform}
\newacronym{ISTRT}{ISTRT}{Inverse Set Theoretic Rajan Transform}

\newacronym{XML}{XML}{Extensible Markup Language}









% add all entries in the glossary 
% - http://tex.stackexchange.com/questions/150228/glossaries-display-the-page-number-but-not-the-current-page 
%\glsaddallunused