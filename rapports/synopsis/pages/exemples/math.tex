

\section{Maths}
% http://dalissier.perso.math.cnrs.fr/pdf/latex/chapitre4.pdf

\subsection{environnements}

\subsubsection{en ligne}

En lignes : $math$

Encore en ligne : 
\begin{math}
	math
\end{math}


Si l'équation est coupée entre deux lignes, utiliser "mbox" \mbox{$f(x)=12$}


\subsubsection{en centrée}

Équation centrée : 

\begin{displaymath}
	math centree
\end{displaymath}


Équation centrée et numérotée :

\begin{equation}
	\label{eq:equation}
	math centree
\end{equation}


Utilisation des références possibles : \ref{eq:equation}.


\subsection{opérations possibles}

\subsubsection{indice}

$variable_{indice}$

\subsubsection{exposant}

$variable^{exposant}$

\subsubsection{fraction}

$\frac{nominateur}{denominateur}$

\subsubsection{racine}

racine : $\sqrt{equation}$

racine n-ieme : $\sqrt[n]{equation}$

\subsubsection{Opérateurs}


\begin{itemize}
	\item somme $\sum$
	\item produit $\prod$
	\item intégrale $\int$
	\item grande intersection $\bigcap$
	\item grande union $\bigcup$
	\item intersection $\cap$
	\item union $\cup$
	\item diviser $\div$
	\item multiplier $\times$
	\item point $\cdot$
	\item $\backslash$
\end{itemize}

Opérateurs binaires $\leq \le \geq \ge \equiv \sim \simeq \approx \perp \subset \in \not\in \not\subset \not= \not\equiv$

Un exemple : $resultat = \sum_{x=0}^{10} x$


$\displaystyle\sum\nolimits_{i=1}^n $

vecteur : $\overrightarrow{a}$

barre : $\overline{a}$

n fois : $\underbrace{a \times \dots \times a}_{n fois}$

grandes parenthèses $\left( x + \frac{1}{2} \right)$

marches avec tous les autres aussi $\left\{ \left [ \left(  \right) \right] \right\}$


\subsection{fonctions mathématiques}

\begin{itemize}
	\item cos : $\cos x$
	\item sin : $\sin x$
	\item tan : $\tan x$
	\item lim : $\lim$
	\item min : $\min$
	\item ln : $\ln $
	\item exp : $\exp x$
	\item log : $\log x$
	\item arg : $\arg x$
	\item max : $\max$
	\item modulo : $a \bmod b$
\end{itemize}


\subsection{symboles}

\subsubsection{les formats}

Caractères 
\begin{itemize}
	\item $\mathrm{romain}$
	\item $\mathit{italique}$
	\item $\mathbf{gras}$
	\item $\mathcal{ABCDEFGHIJKLMNOPQRSTUVWXYZ}$
	\item 
$\mathbb{ABCDEFGHIJKLMNOPQRSTUVWXYZ}$

\end{itemize}


\subsubsection{lettres grecques}

$\alpha$ $\beta$ $\gamma$ $\delta$ $\epsilon$ $\varepsilon$ $\theta$ $\lambda$ $\mu$ $\pi$ $\rho$ $\sigma$ $\phi$ $\varphi$ $\psi$ $\omega$ $\Gamma$ $\Sigma$ $\Psi$ $\Delta$ $\Omega$ $\Pi$ $\Phi$



\subsubsection{autres symboles}

\begin{itemize}
	\item petits points  $\dots$
	\item flèche droite  $\rightarrow$
	\item infini $\infty$
	\item cercle $\circ$
	\item vide $\emptyset$
	\item accolade $\lbrace \rbrace$
\end{itemize}

\subsection{tableaux et matrices}

\begin{equation}
	\begin{array}{lcc}
	x & = & 3 \\
	y & : & 4	
	\end{array}
\end{equation}
