

% \newglossaryentry{<label>}{
%	 name = {<nom>},
%	 description = {<description>},
%	 first = {<premier appel>},
%	 plural = {<si on ajoute pas de 's' à la fin>},
%	 symbol = {<symbole pi>},
% 	 see = {<label_a_voir},
% }

% \newacronym{<label>}{<abbreviation>}{<full>}
% \newacronym[longplural={pluriel}]{<label>}{<abbreviation>}{<full>}


% utilisation (pour tout):
% - \gls(<label>)
% - \Gls(<label>)  		=> avec majuscule
% - \glspl(<label>)  	=> pluriel
% - \Glspl(<label>)		=> pluriel avec majuscule
% - \glsdesc{<label>}  	=> affiche la description
% - \glssymbol{<label>}	=> affiche le symbole

% utilisation en plus pour les accronymes : 
% - \acrshort{<label>}	=> donne l'abbréviation
% - \acrlong{<label>}	=> donne la description "full"
% - \acrfull{<label>}	=> donne "full (abbreviation)"


\newglossaryentry{CLAP}{
	 name = {CLAP},
	 description = {Cellular Logic Array Processing},
	 first = {Cellular Logic Array Processing (CLAP)},
}

\newglossaryentry{DEM}{
	 name = {DEM},
	 description = {Digital Elevation Model},
	 first = {Digital Elevation Model (DEM)},
}

\newglossaryentry{FT}{
	 name = {FT},
	 description = {Fourier Transform},
	 first = {Fourier Transform (FT)},
}

\newglossaryentry{GUI}{
	 name = {GUI},
	 description = {Graphic User Interface},
	 first = {Graphic User Interface (GUI)},
}

\newglossaryentry{RT}{
	 name = {RT},
	 description = {Rajan Transform},
	 first = {Rajan Transform (RT)},
}

\newglossaryentry{STRT}{
	 name = {STRT},
	 description = {Set Theoretic Rajan Transform},
	 first = {Set Theoretic Rajan Transform (STRT)},
}

\newglossaryentry{ISTRT}{
	 name = {ISTRT},
	 description = {Inverse Set Theoric Rajan Transform},
	 first = {Inverse Set Theoric Rajan Transform (ISTRT)},
}

\newglossaryentry{PSO}{
	 name = {PSO},
	 description = {Particle Swarm Optimisation},
	 first = {Particle Swarm Optimisation (PSO)},
}